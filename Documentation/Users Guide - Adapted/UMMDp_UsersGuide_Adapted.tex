\documentclass[11pt,a4paper,twoside,final,onecolumn,titlepage]{article}

% PACKAGES
\usepackage[utf8]{inputenc}
\usepackage[english]{babel}
\usepackage{amsmath}
\usepackage{amsfonts}
\usepackage{amssymb}
\usepackage{bm}
\usepackage{hyperref}
\usepackage{xcolor}
\usepackage[T1]{fontenc}
\hypersetup{colorlinks,citecolor=black,filecolor=black,linkcolor=black,urlcolor=black}
\usepackage{tikz}
\usetikzlibrary{shapes}

% -------------------------------------------------------
% Preâmbulo do documento - Margens e Mancha Gráfica
% -------------------------------------------------------
\setlength{\textwidth}{150mm}          % Largura da Mancha Gráfica
\setlength{\textheight}{225mm}        % Altura da Mancha Gráfica
\setlength{\topmargin}{0mm}             % Margem de Topo (default=1in)
\setlength\oddsidemargin{8mm}    % Margem Direita Páginas Direitas
\setlength\evensidemargin{8mm}   % Margem Esquerda Páginas Esquerdas

\definecolor{twitblue}{HTML}{1DA1F2}
\newcommand{\verified}[1][1]{%
  \begin{tikzpicture}[scale=#1]%
    \node [draw,fill,black,cloud,cloud puffs=8,cloud puff arc=135, inner sep={#1*0.4ex}] {};
    \draw [white,scale=0.13,line cap=round,line width={#1*0.2mm},line join=round](-.4,-.05) -- (-.1,-.3) -- (.4,.4);
  \end{tikzpicture}}
\begin{document}

\begin{titlepage}
\topskip0pt
\LARGE JANCAE

\large Japan Association for Nonlinear CAE

\vspace*{150pt}
\Huge \textbf{UMMDp}

\vspace*{10pt}
\LARGE Unified Material Model Driver for Plasticity

\vspace*{50pt}
\begin{flushright}
\Huge \textbf{User's Guide}

\LARGE Abaqus
\end{flushright}

\vspace*{50pt}
\begin{center}

\vspace*{180pt}
\large Version 2.0

\large Adapted from original document.

\end{center}
\end{titlepage}

\tableofcontents

\newpage
\section{Preface}

The Unified Material Model Driver for Plasticity (UMMDp) library is distributed as Fortran source codes. The Fortran compiler specified by the software vender in your analysis environment must be prepared. Please see the manual of Abaqus for the details of the environment for compiling the user subroutines. 

In this guide, the sections of Abaqus related to the use of the UMMDp are described below, as well as the details on how to use the UMMDp with Abaqus. In the following procedures, the prompts “\%” and “\textgreater ” represent the examples in UNIX/Linux and Windows, respectively. The common steps for using UMMDp with Abaqus are:

\begin{enumerate}
	\item Preparation of UMMDp source files.
	\begin{itemize}
     	\item[$-$] Merge UMMDp source files into one file.
  	\end{itemize}
  	\item Writing procedure to call UMMDp in the input data.
  	\begin{itemize}
     	\item[$-$] To call the material user subroutine, specific keywords are required to be written in the input data. The keywords include the material constants, such as the coefficients of the yield function.
  	\end{itemize}
  	\item Execution of Abaqus software with the UMMDp.
  	\begin{itemize}
     	\item[$-$] When a command is typed to execute Abaqus, options are added to compile the UMMDp and to link it to Abaqus.
  	\end{itemize}
\end{enumerate}

\newpage
\section{Abaqus}

\subsection{Related Section of User's Manual}

\begin{enumerate}
	\item Command to execute with user-subroutine
	 \begin{itemize}
     	\item Abaqus Analysis User's Guide
     	\begin{itemize}
     		\item[$\circ$] 3.2.2 Abaqus/Standard, Abaqus/Explicit, and Abaqus/CFD execution.
		\end{itemize}     	 
  	\end{itemize}
  	\item Option for execution without compile
  	\begin{itemize}
     	\item Abaqus Analysis User's Guide
     	\begin{itemize}
     		\item[$\circ$] 3.2.18 Making user-defined executables and subroutines.
		\end{itemize}     
  	\end{itemize}
  	 \item Keywords for setup of the UMMDp
  	\begin{itemize}
     	\item Abaqus Keywords Reference Guide
     	\begin{itemize}
     		\item[$\circ$] *DEPVAR: define the number of solution-dependent state variables.
     		\item[$\circ$] *ORIENTATION: define local material axis for anisotropy.
     		\item[$\circ$] *USER MATERIAL: define the material constants used in UMAT.
     		\item[$\circ$] *USER OUTPUT VARIABLE: define the number of user output variables.
     	\end{itemize}
  	\end{itemize}
  	 \item Specification of Abaqus user subroutines used in the UMMDp
  	\begin{itemize}
     	\item Abaqus User Subroutines Reference Guide
     	\begin{itemize}
     		\item[$\circ$] 1.1.44 UMAT: user subroutine to define a material's mechanical behavior.
     		\item[$\circ$] 1.1.58 UVARM: user subroutine to generate element output.
     	\end{itemize}
  	\end{itemize}
  	 \item User defined mechanical properties with UMAT
  	 \begin{itemize}
    	 \item Abaqus Analysis User's Guide
     	\begin{itemize}
     		\item[$\circ$] 26.7.1 User-defined mechanical material behavior.
     	\end{itemize}
  	\end{itemize}
\end{enumerate}

\subsection{Usage}

\subsubsection{Preparation of Program Source Files}
Concatenate the UMMDp source files into one single file with the “plug-in” file first.
		\begin{itemize}
			\item Unix/Linux\\
			\par
			\texttt{\fbox{
				\begin{minipage}{0.9\textwidth}
					\% \, cd \, dir\_ummdp\\
					\% \, cp \, plug\_ummdp\_abaqus.f \, jobname\_ummdp.f\\
					\% \, cat \, ummdp*.f \textgreater\!\textgreater \,jobname\_ummdp.f
				\end{minipage}
			}}
		\end{itemize}
		\par\medskip
		\begin{itemize}
			\item Windows\\
			\par
			\texttt{\fbox{
				\begin{minipage}{0.9\textwidth}
					\textgreater \, cd \, dir\_ummdp\\
					\textgreater \, copy \, plug\_ummdp\_abaqus.f \, jobname\_ummdp.f\\
					\textgreater \, type \, ummdp*.f \textgreater\!\textgreater  \,jobname\_ummdp.f
				\end{minipage}
			}}
		\end{itemize}
	
\subsubsection{Preparation of the Input File}
This section describes the keywords in the input data file for use in the UMMDp.

\begin{enumerate}
		\item Definition of the principal axis for the material anisotropy (refer to the manual)
		\par\bigskip	
		\texttt{\fbox{
		\begin{minipage}{0.9\textwidth}
			*ORIENTATION, NAME=ORI-1\\
			1., 0., 0., 0., 1., 0.\\                                                                                                                                                                           
			3, 0.  
		\end{minipage}
		}}
		\par\bigskip
\end{enumerate}

\begin{enumerate}
		\item[2.] Definition of the material model (the details will be provided later)\\
		\par
		\texttt{\fbox{
		\begin{minipage}{0.9\textwidth}
			*MATERIAL, NAME=UMMDp\\
			*USER MATERIAL, CONSTANTS=27\\
			0, 0, 1000.0, 0.3, 2, -0.069, 0.936, 0.079,\\
			1.003, 0.524, 1.363, 0.954, 1.023, 1.069, 0.981, 0.476,\\
			0.575, 0.866, 1.145, -0.079, 1.404, 1.051, 1.147, 8.0,\\
			0, 1.0, 0
		\end{minipage}
		}}
		\par\bigskip
\end{enumerate}

\begin{enumerate}		
		\item[3.] Define the number of internal state variables (SDV)
		\item[] Set the number of state variables to 1+NTENS, where NTENS is the number of components of the tensor variables. NTENS=3 for plane stress or a shell element, and NTENS=6 for a solid element. The 1st “1” is reserved for the equivalent plastic strain, and NTENS is reserved for the plastic strain components. The following example corresponds to a solid element without kinematic hardening:\\
		\par
		\texttt{\fbox{
		\begin{minipage}{0.9\textwidth}
			*DEPVAR\\
			7,
		\end{minipage}
		}}	
		\par\bigskip
		\item[] In the case of kinematic hardening, the number of internal state variables corresponds to the equivalent plastic strain, plastic strain components and components of each partial back-stress tensor. 
		\par\bigskip
\end{enumerate}

\begin{enumerate}
	\item[4.] Define the user output variables (UVARM)
	\item[] UMMDp can output the following three user output variables:
	\begin{enumerate}
		\item current equivalent stress (the value calculated by substituting the stress components for the yield function).
		\item current yield stress (the value calculated by substituting the equivalent plastic strain for the function of the isotropic hardening curve).
		\item current components of the total back-stress tensor.\\		
	\end{enumerate}
		\par
		\texttt{\fbox{
		\begin{minipage}{0.9\textwidth}
			*USER OUTPUT VARIABLES\\
			8,
		\end{minipage}
		}}	
		\par\bigskip
\end{enumerate}

\begin{enumerate}
	\item[5.] Define output variables for post processing
	\item[] This keyword controls the output variables (e.g., equivalent plastic strain and equivalent stress) for post processing.\\
		\par
		\texttt{\fbox{
		\begin{minipage}{0.9\textwidth}
			*OUTPUT, FIELD\\
			*ELEMENT OUTPUT\\
			SDV, UVARM
		\end{minipage}
		}}	
		\par\bigskip
\end{enumerate}

\subsubsection{Execution of Program}
To execute the program there are two options: (a) link the user subroutine in source code and (b) link the user subroutine previously compiled.

\begin{enumerate}
	\item[(a)] To execute the program with the user subroutine in source code, execute the command:\\
	\par
	\texttt{\fbox{
	\begin{minipage}{0.9\textwidth}
		\%\textgreater \, abaqus job=jobname user=jobname\_ummdp.f
	\end{minipage}
	}}	
	\par\bigskip
\end{enumerate}

\begin{enumerate}
\item[(b)] To execute the program with the user subroutine previously compiled, execute the commands:\\
	\par
	\texttt{\fbox{
	\begin{minipage}{0.9\textwidth}
		\textgreater \, abaqus job=jobname user=jobname\_ummdp.obj
	\end{minipage}
	}}	
	\par\bigskip
	\par
	\texttt{\fbox{
	\begin{minipage}{0.9\textwidth}
		\% \,abaqus job=jobname user=jobname\_ummdp.o
	\end{minipage}
	}}	
	\par\bigskip
\item[] The command that compiles the file \texttt{jobname\_ummdp.obj/o} is:\\
	\par
	\texttt{\fbox{
	\begin{minipage}{0.9\textwidth}
		\%\textgreater \, abaqus make library=jobname\_ummdp.f
	\end{minipage}
	}}	
	\par\bigskip
\end{enumerate}
	
	
\newpage
\section{Setup Data Input}

The input data in UMMDp is defined as follows:

\begin{enumerate}
	\item Parameter for Debug and Print
	\item	Parameters for Elastic Properties
	\item Parameters for Yield Function
	\item Parameters for Isotropic Hardening
	\item Parameters for Kinematic Hardening
\end{enumerate}

\noindent The detail of data is shown as follows and in addition, an example of input data is described.

\subsection{Options for Debug and Print}

The first input parameter corresponds to the definition of debug and print mode, defined by the variable \texttt{nvbs0}. It is a mandatory parameter and the options are:

\begin{itemize}
	\item 0 - Error messages only
	\item 1 - Summary of Multistage Return Mapping
	\item 2 - Detail of Multistage Return Mapping and summary of Newton-Raphson
	\item 3 - Detail of Newton-Raphson
	\item 4 - Input/Output
	\item 5 - All status for debug and print
\end{itemize}

\subsection{Options for Elastic Properties}

\begin{itemize}
	\item \texttt{prela(1)} - ID for elastic properties
	\item \texttt{prela(2$\mathtt{\sim}$)} - Data depends on ID
\end{itemize}

\noindent Only isotropic Hooke elastic properties can be defined. There are 2 ways to set them: 

\begin{itemize}
	\item Young's Modulus and Poisson’s Ratio
	\begin{itemize}
		\item[$\circ$] ID = 0
		\item[$\circ$] \texttt{prela(1) = 0}
		\item[$\circ$] \texttt{prela(2) = 200.0E+3} (Young’s modulus \textit{E})
		\item[$\circ$] \texttt{prela(3) = 0.3} (Poisson's ratio $\nu$)
	\end{itemize}
\end{itemize}

\begin{itemize}
	\item Bulk Modulus and Modulus of Rigidity
	\begin{itemize}
		\item[$\circ$] ID = 1
		\item[$\circ$] \texttt{prela(1) = 0}
		\item[$\circ$] \texttt{prela(2) = 166666.7} (Bulk modulus $K=E(1-2\nu)/3$))
		\item[$\circ$] \texttt{prela(3) = 76923.08} (Modulus of rigidity $G=E(1+\nu)/2$)
	\end{itemize}
\end{itemize}

\subsection{Options for Yield Criterion}

\begin{itemize}
	\item \texttt{pryld(1)} - ID for yield function
	\begin{itemize}
	\item[] (Negative value specifies plane stress yield function)
	\end{itemize}
	\item \texttt{pryld(2$\mathtt{\sim}$)} - Data depends on ID
\end{itemize}

\noindent ID for yield criteria and original papers are introduced. Please refer to the original papers for more detail on the formulation and parameters.

\begin{itemize}
	\item von Mises Isotropic (1913)\footnote{von Mises, R. (1913). Mechanik der festen Korper im plastisch deformablen Zustand. Gottin. Nachr. Math. Phys., 1: 582–592.}\,\,\,\,\,\verified{} 
	\begin{itemize}
		\item[•] ID $= 0$ \hspace{100pt}	\# no subsequent data
		\item[$\circ$] \texttt{pryld(1) = 0}\\
	\end{itemize}
\end{itemize}

\begin{itemize}
	\item Hill Quadratic (1948)\footnote{Hill, R. (1948). A theory of the yielding and plastic flow of anisotropic metals. Proc. Roy. Soc. London, 193:281–297.}\,\,\,\,\,\verified{} 
	\begin{itemize}
		\item[•] ID $= 1$ \hspace{100pt}	\# of subsequent data: 6
		\item[$\circ$] \texttt{pryld(1)\,\,\,\,\,\,\,\,\,= 1}
		\item[$\circ$] \texttt{pryld(1+1) =} F
		\item[$\circ$] \texttt{pryld(1+2) =} G
		\item[$\circ$] \texttt{pryld(1+3) =} H
		\item[$\circ$] \texttt{pryld(1+4) =} L
		\item[$\circ$] \texttt{pryld(1+5) =} M
		\item[$\circ$] \texttt{pryld(1+6) =} N
	\end{itemize}
	\item[] Note: The parameters (FGHLMN) are the same as Hill's original paper. When F=G=H=1 and L=M=N=3, Hill's function is identical to von Mises.\\
\end{itemize}

\begin{itemize}
	\item Barlat Yld2004-18p (2005)\footnote{Barlat, F., Aretz, H., Yoon, J.W., Karabin, M.E., Brem, J.C., Dick, R.E. (2005). Linear transformation-based anisotropic yield functions., Int. J. Plast. 21:1009–1039}\,\,\,\,\,\verified{} 
	\begin{itemize}
		\item[•] ID $= 2$ \hspace{100pt}	\# of subsequent data: 19
		\item[$\circ$] \texttt{pryld(1)\,\,\,\,\,\,\,\,\,\,\,\,= 2}
		\item[$\circ$] \texttt{pryld(1+1)\,\,\,\,\,\,= $c'_{12}$}
		\item[$\circ$] \texttt{pryld(1+2)\,\,\,\,\,\,= $c'_{13}$}
		\item[$\circ$] \texttt{pryld(1+3)\,\,\,\,\,\,= $c'_{21}$}
		\item[$\circ$] \texttt{pryld(1+4)\,\,\,\,\,\,= $c'_{23}$}
		\item[$\circ$] \texttt{pryld(1+5)\,\,\,\,\,\,= $c'_{31}$}
		\item[$\circ$] \texttt{pryld(1+6)\,\,\,\,\,\,= $c'_{32}$}
		\item[$\circ$] \texttt{pryld(1+7)\,\,\,\,\,\,= $c'_{44}$}
		\item[$\circ$] \texttt{pryld(1+8)\,\,\,\,\,\,= $c'_{55}$}
		\item[$\circ$] \texttt{pryld(1+9)\,\,\,\,\,\,= $c'_{66}$}
		\item[$\circ$] \texttt{pryld(1+10) = $c''_{12}$}
		\item[$\circ$] \texttt{pryld(1+11) = $c''_{13}$}
		\item[$\circ$] \texttt{pryld(1+12) = $c''_{21}$}
		\item[$\circ$] \texttt{pryld(1+13) = $c''_{23}$}
		\item[$\circ$] \texttt{pryld(1+14) = $c''_{31}$}
		\item[$\circ$] \texttt{pryld(1+15) = $c''_{32}$}
		\item[$\circ$] \texttt{pryld(1+16) = $c''_{44}$}
		\item[$\circ$] \texttt{pryld(1+17) = $c''_{55}$}
		\item[$\circ$] \texttt{pryld(1+18) = $c'_{66}$}
		\item[$\circ$] \texttt{pryld(1+19) = $a$} (exponent)
	\end{itemize}
	\item[] Note: The order of parameters given as input is the same as in the original paper.\\
\end{itemize}

\begin{itemize}
	\item Cazacu (2006)\footnote{Cazacu, O., Plunkett, B., Barlat, F., (2006). Orthotropic yield criterion for hexagonal close packed metals., Int. J. Plasticity 22: 1171–1194.}\,\,\,\,\,\verified{}
	\begin{itemize}
		\item[•] ID $= 3$ \hspace{100pt}	\# of subsequent data: 14
		\item[$\circ$] \texttt{pryld(1)\,\,\,\,\,\,\,\,\,\,\,\,= 3}
		\item[$\circ$] \texttt{pryld(1+1)\,\,\,\,\,\,= $C_{11}$}
		\item[$\circ$] \texttt{pryld(1+2)\,\,\,\,\,\,= $C_{12}$}
		\item[$\circ$] \texttt{pryld(1+3)\,\,\,\,\,\,= $C_{13}$}
		\item[$\circ$] \texttt{pryld(1+4)\,\,\,\,\,\,= $C_{21}$}
		\item[$\circ$] \texttt{pryld(1+5)\,\,\,\,\,\,= $C_{22}$}
		\item[$\circ$] \texttt{pryld(1+6)\,\,\,\,\,\,= $C_{23}$}
		\item[$\circ$] \texttt{pryld(1+7)\,\,\,\,\,\,= $C_{31}$}
		\item[$\circ$] \texttt{pryld(1+8)\,\,\,\,\,\,= $C_{32}$}
		\item[$\circ$] \texttt{pryld(1+9)\,\,\,\,\,\,= $C_{33}$}
		\item[$\circ$] \texttt{pryld(1+10) = $C_{44}$}
		\item[$\circ$] \texttt{pryld(1+11) = $C_{55}$}
		\item[$\circ$] \texttt{pryld(1+12) = $C_{66}$}
		\item[$\circ$] \texttt{pryld(1+13) = $a$} (exponent) 
		\item[$\circ$] \texttt{pryld(1+14) = $k$} (tension-compression ratio)\\
	\end{itemize}
\end{itemize}

\begin{itemize}
	\item Karafillis-Boyce (1993)\footnote{Karafillis, A. P.; Boyce, M. C. (1993). A general anisotropic yield criterion using bounds and a transformation weighting tensor., J. Mech. Phys. Solids, 41:1859-1886.}
	\begin{itemize}
		\item[•] ID $= 4$ \hspace{100pt}	\# of subsequent data: 8
		\item[$\circ$] \texttt{pryld(1)\,\,\,\,\,\,\,\,\,= 4}
		\item[$\circ$] \texttt{pryld(1+1) = $C$}
		\item[$\circ$] \texttt{pryld(1+2) = $\alpha_1$}
		\item[$\circ$] \texttt{pryld(1+3) = $\alpha_2$}
		\item[$\circ$] \texttt{pryld(1+4) = $\gamma_1$}
		\item[$\circ$] \texttt{pryld(1+5) = $\gamma_2$}
		\item[$\circ$] \texttt{pryld(1+6) = $\gamma_3$}
		\item[$\circ$] \texttt{pryld(1+7) = $c$}
		\item[$\circ$] \texttt{pryld(1+8) = $k$} ($k$ of exponent $2k$)\\
	\end{itemize}
\end{itemize}

\begin{itemize}
	\item Hu (2005)\footnote{Hu, W. (2005). An orthotropic yield criterion in a 3-D general stress state. Int. J. Plasticity 21:1771–1796.}
	\begin{itemize}
		\item[•] ID $= 5$	\hspace{100pt}	\# of subsequent data: 10
		\item[$\circ$] \texttt{pryld(1)\,\,\,\,\,\,\,\,\,\,\,\,= 5}
		\item[$\circ$] \texttt{pryld(1+1)\,\,\,\,\,\,= $X_1$}
		\item[$\circ$] \texttt{pryld(1+2)\,\,\,\,\,\,= $X_2$}
		\item[$\circ$] \texttt{pryld(1+3)\,\,\,\,\,\,= $X_3$}
		\item[$\circ$] \texttt{pryld(1+4)\,\,\,\,\,\,= $X_4$}
		\item[$\circ$] \texttt{pryld(1+5)\,\,\,\,\,\,= $X_5$}
		\item[$\circ$] \texttt{pryld(1+6)\,\,\,\,\,\,= $X_6$}
		\item[$\circ$] \texttt{pryld(1+7)\,\,\,\,\,\,= $X_7$}
		\item[$\circ$] \texttt{pryld(1+8)\,\,\,\,\,\,= $C_1$}
		\item[$\circ$] \texttt{pryld(1+9)\,\,\,\,\,\,= $C_2$}
		\item[$\circ$] \texttt{pryld(1+10) = $C_3$}\\
	\end{itemize}
\end{itemize}

\begin{itemize}
	\item Yoshida 6th Polynomial (2011)\footnote{Yoshida, F., Hamasaki, H., Uemori, T. (2013). A user-friendly 3D yield function to describe anisotropy of steel sheets. Int. J. Plasticity, 45:119-139.}
	\begin{itemize}
		\item[•] ID $= 6$	\hspace{100pt}	\# of subsequent data: 16
		\item[$\circ$] \texttt{pryld(1)\,\,\,\,\,\,\,\,\,\,\,\,= 6}
		\item[$\circ$] \texttt{pryld(1+1)\,\,\,\,\,\,= $C_1$}
		\item[$\circ$] \texttt{pryld(1+2)\,\,\,\,\,\,= $C_2$}
		\item[$\circ$] \texttt{pryld(1+3)\,\,\,\,\,\,= $C_3$}
		\item[$\circ$] \texttt{pryld(1+4)\,\,\,\,\,\,= $C_4$}
		\item[$\circ$] \texttt{pryld(1+5)\,\,\,\,\,\,= $C_5$}
		\item[$\circ$] \texttt{pryld(1+6)\,\,\,\,\,\,= $C_6$}
		\item[$\circ$] \texttt{pryld(1+7)\,\,\,\,\,\,= $C_7$}
		\item[$\circ$] \texttt{pryld(1+8)\,\,\,\,\,\,= $C_8$}
		\item[$\circ$] \texttt{pryld(1+9)\,\,\,\,\,\,= $C_9$}
		\item[$\circ$] \texttt{pryld(1+10) = $C_{10}$}
		\item[$\circ$] \texttt{pryld(1+11) = $C_{11}$}
		\item[$\circ$] \texttt{pryld(1+12) = $C_{12}$}
		\item[$\circ$] \texttt{pryld(1+13) = $C_{13}$}
		\item[$\circ$] \texttt{pryld(1+14) = $C_{14}$}
		\item[$\circ$] \texttt{pryld(1+15) = $C_{15}$}
		\item[$\circ$] \texttt{pryld(1+16) = $C_{16}$}\\
	\end{itemize}
\end{itemize}

\begin{itemize}
	\item Gotoh Biquadratic (1978)\footnote{Gotoh, M., (1977). A theory of plastic anisotropy based on a yield function of fourth order (plane stress state) - I, Int. J. Mech. Sci. , 19-9 : 505-512. (see also J.JSTP, 19(1978) :337-385 )}
	\begin{itemize}
		\item[•] ID $= -1$	 \hspace{98pt} \# of subsequent data: 9
		\item[$\circ$] \texttt{pryld(1)\,\,\,\,\,\,\,\,\,= -1}
		\item[$\circ$] \texttt{pryld(1+1) = $A_1$}
		\item[$\circ$] \texttt{pryld(1+2) = $A_2$}
		\item[$\circ$] \texttt{pryld(1+3) = $A_3$}
		\item[$\circ$] \texttt{pryld(1+4) = $A_4$}
		\item[$\circ$] \texttt{pryld(1+5) = $A_5$}
		\item[$\circ$] \texttt{pryld(1+6) = $A_6$}
		\item[$\circ$] \texttt{pryld(1+7) = $A_7$}
		\item[$\circ$] \texttt{pryld(1+8) = $A_8$}
		\item[$\circ$] \texttt{pryld(1+9) = $A_9$}\\
	\end{itemize}
\end{itemize}

\begin{itemize}
	\item Barlat Yld2000-2d (2003)\footnote{Barlat, F., Brem, J.C., Yoon, J.W., Chung, K., Dick, R.E., Lege, D.J., Pourboghrat, F., Choi, S.H., Chu, E. (2003). Plane stress yield function for aluminium alloy sheets-part 1: theory, Int. J. Plasticity, 19:1297-1319.}\,\,\,\,\,\verified{} 
	\begin{itemize}
		\item[•] ID $= -2$ \hspace{98pt} \# of subsequent data: 9
		\item[$\circ$] \texttt{pryld(1)\,\,\,\,\,\,\,\,\,= -2}
		\item[$\circ$] \texttt{pryld(1+1) = $\alpha_1$}
		\item[$\circ$] \texttt{pryld(1+2) = $\alpha_2$}
		\item[$\circ$] \texttt{pryld(1+3) = $\alpha_3$}
		\item[$\circ$] \texttt{pryld(1+4) = $\alpha_4$}
		\item[$\circ$] \texttt{pryld(1+5) = $\alpha_5$}
		\item[$\circ$] \texttt{pryld(1+6) = $\alpha_6$}
		\item[$\circ$] \texttt{pryld(1+7) = $\alpha_7$}
		\item[$\circ$] \texttt{pryld(1+8) = $\alpha_8$}
		\item[$\circ$] \texttt{pryld(1+9) = $a$} (exponent)\\
	\end{itemize}
\end{itemize}

\begin{itemize}
	\item Vegter (2006)\footnote{Vegter, H., den Boogaard, A.H. van (2006). A plane stress yield function for anisotropic sheet material by interpolation of biaxial stress states, Int. J. Plasticity, 22:557-580.}
	\begin{itemize}
		\item[•] ID $= -3$ \hspace{98pt} \# of subsequent data: $3+4n$
		\item[$\circ$] \texttt{pryld(1)\,\,\,\,\,\,\,\,\,\,\,\,\,\,\,\,\,\,\,\,\,\,\,\,\,\,\,\,\,\,\,\,\,\,\,\,\,\,\,= -3}
		\item[$\circ$] \texttt{pryld(1+1)\,\,\,\,\,\,\,\,\,\,\,\,\,\,\,\,\,\,\,\,\,\,\,\,\,\,\,\,\,\,\,\,\,= n} (max of $i$)
		\item[$\circ$] \texttt{pryld(1+2)\,\,\,\,\,\,\,\,\,\,\,\,\,\,\,\,\,\,\,\,\,\,\,\,\,\,\,\,\,\,\,\,\,= f\_bi0}
		\item[$\circ$] \texttt{pryld(1+3)\,\,\,\,\,\,\,\,\,\,\,\,\,\,\,\,\,\,\,\,\,\,\,\,\,\,\,\,\,\,\,\,\,= r\_bi0}
		\item[$\circ$] \texttt{pryld(1+3+(i-1)*4+1) = phi\_uniaxial(i)}
		\item[$\circ$] \texttt{pryld(1+3+(i-1)*4+2) = phi\_shear(i)}
		\item[$\circ$] \texttt{pryld(1+3+(i-1)*4+3) = ph\_planestrain(i)}
		\item[$\circ$] \texttt{pryld(1+3+(i-1)*4+4) = omega(i)}\\
	\end{itemize}
\end{itemize}

\begin{itemize}
	\item Banabic BBC2005\footnote{Banabic, D., Aretz,, D.S. Comsa, H., Paraianu, L.(2005). An improved analytical description of orthotropy in metallic sheets, Int. J. Plasticity 21:493–512.}
	\begin{itemize}
		\item[•] ID $= -4$ \hspace{98pt} \# of subsequent data: 9
		\item[$\circ$] \texttt{pryld(1)\,\,\,\,\,\,\,\,\,= -4}
		\item[$\circ$] \texttt{pryld(1+1) = $k$} ($k$ of exponent $2k$)
		\item[$\circ$] \texttt{pryld(1+2) = $a$}
		\item[$\circ$] \texttt{pryld(1+3) = $b$}
		\item[$\circ$] \texttt{pryld(1+4) = $L$}
		\item[$\circ$] \texttt{pryld(1+5) = $M$}
		\item[$\circ$] \texttt{pryld(1+6) = $N$}
		\item[$\circ$] \texttt{pryld(1+7) = $P$}
		\item[$\circ$] \texttt{pryld(1+8) = $Q$}
		\item[$\circ$] \texttt{pryld(1+9) = $R$}\\
	\end{itemize}
\end{itemize}

\begin{itemize}
	\item Barlat Yld89\footnote{Barlat, F., Lian, J.(1989). Plastic behavior and stretchability of sheet metals. Part I: a yield function for orthotropic sheets under plane stress conditions. Int. J. Plasticity. 5:51–66.}
	\begin{itemize}
		\item[•] ID $= -5$ \hspace{98pt} \# of subsequent data: 4
		\item[$\circ$] \texttt{pryld(1)\,\,\,\,\,\,\,\,\,= -5}
		\item[$\circ$] \texttt{pryld(1+1) = $M$} (exponent)
		\item[$\circ$] \texttt{pryld(1+2) = $a$}
		\item[$\circ$] \texttt{pryld(1+3) = $h$}
		\item[$\circ$] \texttt{pryld(1+4) = $p$}\\
	\end{itemize}
\end{itemize}

\begin{itemize}
	\item Banabic BBC2008\footnote{Comsa, D.S., Banabic,D. (2008). Plane-stress yield criterion for highly-anisotropic sheet metals, Proc. of NUMISHEET 2008.}
	\begin{itemize}
		\item[•] ID $= -6$ \hspace{98pt} \# of subsequent data: $2+8s$
		\item[$\circ$] \texttt{pryld(1)\,\,\,\,\,\,\,\,\,\,\,\,\,\,\,\,\,\,\,\,\,\,\,\,\,\,\,\,\,\,\,\,\,\,\,\,\,\,\,= -6}
		\item[$\circ$] \texttt{pryld(1+1)\,\,\,\,\,\,\,\,\,\,\,\,\,\,\,\,\,\,\,\,\,\,\,\,\,\,\,\,\,\,\,\,\,= $s$} (max of $i$)
		\item[$\circ$] \texttt{pryld(1+2)\,\,\,\,\,\,\,\,\,\,\,\,\,\,\,\,\,\,\,\,\,\,\,\,\,\,\,\,\,\,\,\,\,= $k$} ($k$ of exponent $2k$)
		\item[$\circ$] \texttt{pryld(1+2+(i-1)*8+1) = $l_1$}
		\item[$\circ$] \texttt{pryld(1+2+(i-1)*8+2) = $l_2$}
		\item[$\circ$] \texttt{pryld(1+2+(i-1)*8+3) = $m_1$}
		\item[$\circ$] \texttt{pryld(1+2+(i-1)*8+4) = $m_2$}
		\item[$\circ$] \texttt{pryld(1+2+(i-1)*8+5) = $m_3$}
		\item[$\circ$] \texttt{pryld(1+2+(i-1)*8+6) = $n_1$}
		\item[$\circ$] \texttt{pryld(1+2+(i-1)*8+7) = $n_2$}
		\item[$\circ$] \texttt{pryld(1+2+(i-1)*8+8) = $n_3$}\\
	\end{itemize}
\end{itemize}

\begin{itemize}
	\item Hill 1990\footnote{Hill, R (1990). Constitutive modelling of orthotropic plasticity in sheet metals, Journal of the Mechanics and Physics of Solids.}
	\begin{itemize}
		\item[•] ID $= -7$ \hspace{98pt} \# of subsequent data: $5$
		\item[$\circ$] \texttt{pryld(1)\,\,\,\,\,\,\,\,\,= -7}
		\item[$\circ$] \texttt{pryld(1+1) = $a$}
		\item[$\circ$] \texttt{pryld(1+2) = $b$}
		\item[$\circ$] \texttt{pryld(1+3) = $\tau$}
		\item[$\circ$] \texttt{pryld(1+4) = $\sigma_{b}$}
		\item[$\circ$] \texttt{pryld(1+5) = $m$}\\
	\end{itemize}
\end{itemize}

\subsection{Options for Isotropic Hardening}

\begin{itemize}
	\item \texttt{prihd(1)} - ID for isotropic hardening
	\item \texttt{prihd(2$\mathtt{\sim}$)} - Data depends on ID
\end{itemize}

\noindent The equation of flow curve is introduced for each law, where $\sigma_{\textrm{y}}$ is the yield stress, $\sigma_{\textrm{y}_0}$ is the initial yield stress and $p$ the equivalent plastic strain.

\begin{itemize}
	\item Perfectly Plastic\,\,\,\,\,\verified{} 
	\begin{itemize}
		\item[•] ID $=0$ \hspace{100pt} \# of subsequent data: 1
		\item[$\circ$] \texttt{prihd(1)\,\,\,\,\,\,\,\,\,= 0}
		\item[$\circ$] \texttt{prihd(1+1) = $\sigma_{y}$}\\
	\end{itemize}
\end{itemize}

\begin{itemize}
	\item Linear Hardening: $\displaystyle \sigma_{\textrm{y}} = \sigma_{\textrm{y}_0} + Hp$\,\,\,\,\,\verified{} 
	\begin{itemize}
		\item[•] ID $=1$ \hspace{100pt} \# of subsequent data: 2
		\item[$\circ$] \texttt{prihd(1)\,\,\,\,\,\,\,\,\,= 1}
		\item[$\circ$] \texttt{prihd(1+1) = $\sigma_{\textrm{y}_0} $}
		\item[$\circ$] \texttt{prihd(1+2) = $H$}\\
	\end{itemize}
\end{itemize}

\begin{itemize}
	\item Swift: $\displaystyle \sigma_{\textrm{y}} = K\left(\epsilon_{0}+p\right)^n$\,\,\,\,\,\verified{} 
	\begin{itemize}
		\item[•] ID $=2$ \hspace{100pt} \# of subsequent data: 3
		\item[$\circ$] \texttt{prihd(1)\,\,\,\,\,\,\,\,\,= 2}
		\item[$\circ$] \texttt{prihd(1+1) = $K$} 
		\item[$\circ$] \texttt{prihd(1+2) = $\epsilon_0$} 
		\item[$\circ$] \texttt{prihd(1+3) = $n$}\\
	\end{itemize}
\end{itemize}

\begin{itemize}
	\item Ludwick: $\displaystyle \sigma_{\textrm{y}} = \sigma_{\textrm{y}_0} +cp^n$\,\,\,\,\,\verified{} 
	\begin{itemize}
		\item[•] ID $=3$ \hspace{100pt} \# of subsequent data: 3
		\item[$\circ$] \texttt{prihd(1)\,\,\,\,\,\,\,\,\,= 3}
		\item[$\circ$] \texttt{prihd(1+1) = $\sigma_{\textrm{y}_0} $}
		\item[$\circ$] \texttt{prihd(1+2) = $c$} 
		\item[$\circ$] \texttt{prihd(1+3) = $n$}\\
	\end{itemize}
\end{itemize}

\begin{itemize}
	\item Voce: $\displaystyle \sigma_{\textrm{y}} = \sigma_{\textrm{y}_0} +Q\left(1-\textrm{exp}(-bp)\right)$\,\,\,\,\,\verified{} 
	\begin{itemize}
		\item[•] ID $=4$ \hspace{100pt} \# of subsequent data: 3
		\item[$\circ$] \texttt{prihd(1)\,\,\,\,\,\,\,\,\,= 4}
		\item[$\circ$] \texttt{prihd(1+1) = $\sigma_{\textrm{y}_0} $}
		\item[$\circ$] \texttt{prihd(1+2) = $Q$} 
		\item[$\circ$] \texttt{prihd(1+3) = $b$}\\
	\end{itemize}
\end{itemize}

\begin{itemize}
	\item Voce + Linear: $\displaystyle \sigma_{\textrm{y}} = \sigma_{\textrm{y}_0} +Q\left(1-\textrm{exp}(-bp)\right)+Hp$\,\,\,\,\,\verified{} 
	\begin{itemize}
		\item[•] ID $=5$ \hspace{100pt} \# of subsequent data: 4
		\item[$\circ$] \texttt{prihd(1)\,\,\,\,\,\,\,\,\,= 5}
		\item[$\circ$] \texttt{prihd(1+1) = $\sigma_{\textrm{y}_0} $}
		\item[$\circ$] \texttt{prihd(1+2) = $Q$} 
		\item[$\circ$] \texttt{prihd(1+3) = $b$} 
		\item[$\circ$] \texttt{prihd(1+4) = $H$}\\
	\end{itemize}
\end{itemize}

\begin{itemize}
	\item Voce + Swift: $\displaystyle \sigma_{\textrm{y}} = a\left[\sigma_{\textrm{y}_0} + Q\left(1-\textrm{exp}(-bp)\right)\right]+ (1-a)\left[K(\epsilon_0+p)^n\right]$\,\,\,\,\,\verified{} 
	\begin{itemize}
		\item[•] ID $=6$ \hspace{100pt} \# of subsequent data: 7
		\item[$\circ$] \texttt{prihd(1)\,\,\,\,\,\,\,\,\,= 6}
		\item[$\circ$] \texttt{prihd(1+1) = $a$}
		\item[$\circ$] \texttt{prihd(1+2) = $\sigma_{\textrm{y}_0} $}
		\item[$\circ$] \texttt{prihd(1+3) = $Q$} 
		\item[$\circ$] \texttt{prihd(1+4) = $b$} 
		\item[$\circ$] \texttt{prihd(1+5) = $K$} 
		\item[$\circ$] \texttt{prihd(1+6) = $\epsilon_0$} 
		\item[$\circ$] \texttt{prihd(1+7) = $n$}\\
	\end{itemize}
\end{itemize}

\subsection{Options for Kinematic Hardening}

\begin{itemize}
	\item \texttt{prkin(1)} - ID for kinematic hardening
	\item \texttt{prkin(2$\mathtt{\sim}$)} - Data depends on ID
\end{itemize}

\noindent The equation of backstress is introduced for each law, where $\dot{\bm{\alpha}}$ is the total increment of backstress tensor, $\dot{\bm{\alpha}_i}$ is the a term of the total increment of backstress tensor, $\bm{\alpha}$ is the backstress tensor, $\dot{\bm{\epsilon}}^\textrm{p}$ is the increment of plastic strain tensor and $\dot{p}$ is the increment of equivalent plastic strain.

\begin{itemize}
	\item No Kinematic Hardening\,\,\,\,\,\verified{} 
	\begin{itemize}
		\item[•] ID $=0$ \hspace{100pt} \# no subsequent data
		\item[$\circ$] \texttt{prkin(1) = 0}
	\end{itemize}
\end{itemize}

\begin{itemize}
	\item Prager (1949): $\displaystyle \dot{\bm{\alpha}}=\frac{2}{3}c\dot{\bm{\epsilon}}^\textrm{p}$\,\,\,\,\,\verified{} 
	\begin{itemize}
		\item[•] ID $=1$ \hspace{100pt}  \# of subsequent data: 1
		\item[$\circ$] \texttt{prkin(1)\,\,\,\,\,\,\,\,\,= 1}
		\item[$\circ$] \texttt{prkin(1+1) = $c$}\\
	\end{itemize}
\end{itemize}

\begin{itemize}
	\item Ziegler (1959): $\displaystyle \dot{\bm{\alpha}}=c\left(\bm{\sigma}-\bm{\alpha}\right)\dot{p}$\,\,\,\,\,\verified{} 
	\begin{itemize}
		\item[•] ID $=2$ \hspace{100pt}  \# of subsequent data: 1
		\item[$\circ$] \texttt{prkin(1)\,\,\,\,\,\,\,\,\,= 2}
		\item[$\circ$] \texttt{prkin(1+1) = $c$}\\
	\end{itemize}
\end{itemize}

\begin{itemize}
	\item Armstrong-Frederick (1966): $\displaystyle \dot{\bm{\alpha}}=\frac{2}{3}c\dot{\bm{\epsilon}}^\textrm{p} -\gamma\bm{\alpha}\dot{p}$\,\,\,\,\,\verified{} 
	\begin{itemize}
		\item[•] ID $=3$ \hspace{100pt}  \# of subsequent data: 2
		\item[$\circ$] \texttt{prkin(1)\,\,\,\,\,\,\,\,\,= 3}
		\item[$\circ$] \texttt{prkin(1+1) = $c$}
		\item[$\circ$] \texttt{prkin(1+2) = $\gamma$}\\
	\end{itemize}
\end{itemize}

\begin{itemize}
	\item Chaboche (1979): $\displaystyle \dot{\bm{\alpha}}=\sum_{i=1}^{n}\dot{\bm{\alpha}_i}=\sum_{i=1}^{n}\left(\frac{2}{3}c_i\dot{\bm{\epsilon}}^\textrm{p} -\gamma\bm{\alpha}_i\dot{p}\right)$\,\,\,\,\,\verified{} 
	\begin{itemize}
		\item[•] ID $=4$ \hspace{100pt}  \# of subsequent data: $1+2n$
		\item[$\circ$] \texttt{prkin(1)\,\,\,\,\,\,\,\,\,\,\,\,\,\,\,\,\,\,\,\,\,\,\,\,\,\,\,= 4}
		\item[$\circ$] \texttt{prkin(1+1)\,\,\,\,\,\,\,\,\,\,\,\,\,\,\,\,\,\,\,\,\,= $n$}
		\item[$\circ$] \texttt{prkin(1+1+(i*1)) = $c_i$}
		\item[$\circ$] \texttt{prkin(1+1+(i*2)) = $\gamma_i$}\\
	\end{itemize}
\end{itemize}

\begin{itemize}
	\item Chaboche (1979) - Ziegler Type: $\displaystyle \dot{\bm{\alpha}}=\sum_{i=1}^{n}\dot{\bm{\alpha}_i}=\sum_{i=1}^{n}\left(\frac{c_i}{\gamma_i}\left(\bm{\sigma}-\bm{\alpha}\right)-\gamma\bm{\alpha}_i\right)\dot{p}$\,\,\,\,\,\verified{} 
	\begin{itemize}
		\item[•] ID $=5$ \hspace{100pt}  \# of subsequent data: $1+2n$
		\item[$\circ$] \texttt{prkin(1)\,\,\,\,\,\,\,\,\,\,\,\,\,\,\,\,\,\,\,\,\,\,\,\,\,\,\,= 5}
		\item[$\circ$] \texttt{prkin(1+1)\,\,\,\,\,\,\,\,\,\,\,\,\,\,\,\,\,\,\,\,\,= $n$}
		\item[$\circ$] \texttt{prkin(1+1+(i*1)) = $c_i$}
		\item[$\circ$] \texttt{prkin(1+1+(i*2)) = $\gamma_i$}\\
	\end{itemize}
\end{itemize}

\begin{itemize}
	\item Yoshida-Uemori %$\dot{\bm{\alpha}}=...$
	\begin{itemize}
		\item[•] ID $=6$ \hspace{100pt}  \# of subsequent data: 5
		\item[$\circ$] \texttt{prkin(1)\,\,\,\,\,\,\,\,\,= 5}
		\item[$\circ$] \texttt{prkin(1+1) = $C$}
		\item[$\circ$] \texttt{prkin(1+2) = $Y$}
		\item[$\circ$] \texttt{prkin(1+3) = $a$}
		\item[$\circ$] \texttt{prkin(1+4) = $k$}
		\item[$\circ$] \texttt{prkin(1+5) = $b$}\\	
	\end{itemize}
\end{itemize}

\subsection{Parameters for Uncoupled Rupture Criterion}

\begin{itemize}
	\item \texttt{prrup(1)} - ID for uncoupled rupture criterion 
	\item \texttt{prrup(2$\mathtt{\sim}$)} - Data depends on ID
\end{itemize}

\begin{itemize}
	\item No Uncoupled Rupture Criterion\,\,\,\,\,\verified{} 
	\begin{itemize}
		\item[•] ID $=0$ \hspace{100pt} \# no subsequent data
		\item[$\circ$] \texttt{prkin(1) = 0}
	\end{itemize}
\end{itemize}

\begin{itemize}
	\item Equivalent Plastic Strain: $\displaystyle W = \int_{0}^{\epsilon_f} \dot{p}\,\text{d}t$\,\,\,\,\,\verified{} 
	\begin{itemize}
		\item[•] ID $=1$ \hspace{100pt}  \# of subsequent data: 1
		\item[$\circ$] \texttt{prkin(1)\,\,\,\,\,\,\,\,\,= 1}
		\item[$\circ$] \texttt{prkin(1+1) = $W_{L}$}\\
	\end{itemize}
\end{itemize}

\begin{itemize}
	\item Cockroft and Latham: $\displaystyle W =  \int_{0}^{\epsilon_f}\frac{\sigma_{1}}{\bar{\sigma}}\,\text{d}p$\,\,\,\,\,\verified{} 
	\begin{itemize}
		\item[•] ID $=2$ \hspace{100pt}  \# of subsequent data: 1
		\item[$\circ$] \texttt{prkin(1)\,\,\,\,\,\,\,\,\,= 1}
		\item[$\circ$] \texttt{prkin(1+1) = $W_{L}$}\\
	\end{itemize}
\end{itemize}

\begin{itemize}
	\item Rice and Tracey: $\displaystyle W = \int_{0}^{\epsilon_f} \text{exp}\left(\frac{3}{2}\frac{\sigma_h}{\bar{\sigma}}\right)\,\text{d}p$\,\,\,\,\,\verified{} 
	\begin{itemize}
		\item[•] ID $=3$ \hspace{100pt}  \# of subsequent data: 1
		\item[$\circ$] \texttt{prkin(1)\,\,\,\,\,\,\,\,\,= 1}
		\item[$\circ$] \texttt{prkin(1+1) = $W_{L}$}\\
	\end{itemize}
\end{itemize}

\begin{itemize}
	\item Ayada: $\displaystyle W =  \int_{0}^{\epsilon_f}\frac{\sigma_{h}}{\bar{\sigma}}\,\text{d}p$\,\,\,\,\,\verified{} 
	\begin{itemize}
		\item[•] ID $=4$ \hspace{100pt}  \# of subsequent data: 1
		\item[$\circ$] \texttt{prkin(1)\,\,\,\,\,\,\,\,\,= 1}
		\item[$\circ$] \texttt{prkin(1+1) = $W_{L}$}\\
	\end{itemize}
\end{itemize}

\begin{itemize}
	\item Brozzo: $\displaystyle W =  \int_{0}^{\epsilon_f}\frac{2\sigma_{1}}{3\left(\sigma_{1}-\sigma_{h}\right)}\,\text{d}p$\,\,\,\,\,\verified{} 
	\begin{itemize}
		\item[•] ID $=5$ \hspace{100pt}  \# of subsequent data: 1
		\item[$\circ$] \texttt{prkin(1)\,\,\,\,\,\,\,\,\,= 1}
		\item[$\circ$] \texttt{prkin(1+1) = $W_{L}$}\\
	\end{itemize}
\end{itemize}

\newpage
\section{Example of Data Input}

\begin{itemize}
	\item Elastic Properties: $E$=200 GPa, $\nu$=0.3
	\item Yield Criterion: Barlat Yld2004-18p (coefficients of AA6111-T4 given in the original paper)
	\item Isotropic Hardening: Swift
	\item Kinematic Hardening: Armstrong-Frederick (1966)
	\item Uncoupled Rupture Criterion: Ayada
\end{itemize}

\noindent The material parameters and IDs are given as input to the UMMDp in one dimensional array, named by default \texttt{props} in the program. In the beginning of UMMDp, this array is copied to a new variable \texttt{prop}, and the variable to define debug and print mode is excluded from this new array.

\begin{itemize}
		\item[$\circ$] \texttt{props(1)\,\,\,\,\,\,= \textcolor{red}{0}} \hspace{100pt} Debug and Print ID = 0
		\item[$\circ$] \texttt{props(2)\,\,\,\,\,\,= \textcolor{red}{0}} \hspace{100pt} Elastic Property ID = 0
		\item[$\circ$] \texttt{props(3)\,\,\,\,\,\,= 200000} \hspace{72pt} Young’s Modulus $E$ 
		\item[$\circ$] \texttt{props(4)\,\,\,\,\,\,= 0.3} \hspace{89pt} Poisson’s Ratio $\nu$
		\item[$\circ$] \texttt{props(5)\,\,\,\,\,\,= \textcolor{red}{2}} \hspace{101pt} Yield Criterion: ID = 2 (Barlat Yld2004-18p)
		\item[$\circ$] \texttt{props(6)\,\,\,\,\,\,= 1.241024} \hspace{60pt} $c'_{12}$
		\item[$\circ$] \texttt{props(7)\,\,\,\,\,\,= 1.078271} \hspace{60pt} $c'_{13}$
		\item[$\circ$] \texttt{props(8)\,\,\,\,\,\,= 1.216463} \hspace{60pt} $c'_{21}$
		\item[$\circ$] \texttt{props(9)\,\,\,\,\,\,= 1.223867} \hspace{60pt} $c'_{23}$
		\item[$\circ$] \texttt{props(10) = 1.093105} \hspace{60pt} $c'_{31}$
		\item[$\circ$] \texttt{props(11) = 0.889161} \hspace{60pt} $c'_{32}$
		\item[$\circ$] \texttt{props(12) = 0.501909} \hspace{60pt} $c'_{44}$
		\item[$\circ$] \texttt{props(13) = 0.557173} \hspace{60pt} $c'_{55}$
		\item[$\circ$] \texttt{props(14) = 1.349094} \hspace{60pt} $c'_{66}$
		\item[$\circ$] \texttt{props(15) = 0.775366} \hspace{60pt} $c''_{12}$
		\item[$\circ$] \texttt{props(16) = 0.922743} \hspace{60pt} $c''_{13}$
		\item[$\circ$] \texttt{props(17) = 0.765487} \hspace{60pt} $c''_{21}$
		\item[$\circ$] \texttt{props(18) = 0.793356} \hspace{60pt} $c''_{23}$
		\item[$\circ$] \texttt{props(19) = 0.918689} \hspace{60pt} $c''_{31}$
		\item[$\circ$] \texttt{props(20) = 1.027625} \hspace{60pt} $c''_{32}$
		\item[$\circ$] \texttt{props(21) = 1.115833} \hspace{60pt} $c''_{44}$
		\item[$\circ$] \texttt{props(22) = 1.112273} \hspace{60pt} $c''_{55}$
		\item[$\circ$] \texttt{props(23) = 0.589787} \hspace{60pt} $c''_{66}$
		\item[$\circ$] \texttt{props(24) = 8} \hspace{100pt} $a$
		\item[$\circ$] \texttt{props(25) = \textcolor{red}{2}} \hspace{100pt} Isotropic Hardening: ID = 2 (Swift)
		\item[$\circ$] \texttt{props(26) = 541.0} \hspace{76pt} $K$
		\item[$\circ$] \texttt{props(27) = 0.0036} \hspace{72pt} $\epsilon_0$
		\item[$\circ$] \texttt{props(28) = 0.249} \hspace{78pt} $n$
		\item[$\circ$] \texttt{props(29) = \textcolor{red}{3}} \hspace{101pt} Kin. Hardening :ID = 3 (Armstrong-Frederick)
		\item[$\circ$] \texttt{props(30) = 1018.4245} \hspace{56pt} $c$
		\item[$\circ$] \texttt{props(31) = 22.85} \hspace{78pt} $\gamma$
		\item[$\circ$] \texttt{props(32) = \textcolor{red}{4}} \hspace{101pt} Uncoupled Rupture Criterion: ID = 4 (Ayada)
		\item[$\circ$] \texttt{props(33) = 0.5} \hspace{89pt} $W_{L}$
\end{itemize}

\noindent This \texttt{prop(i)} array is divided into each properties in UMMDp as follow:
\begin{itemize}
	\item \texttt{prela(i)} - Elastic Properties
	\item \texttt{pryld(i)} - Yield Criterion
	\item \texttt{prihd(i)} - Isotropic Hardening
	\item \texttt{prkin(i)} -  Kinematic Hardening
	\item \texttt{prrup(i)} -  Uncoupled Rupture Criterion
\end{itemize}

\noindent The model ID is stored in the top of each of these arrays. Here, it is shown the input example of the material data for the abaqus input file. The red letter indicates ID of each properties.
\par\bigskip
\par\bigskip
\noindent
\hspace{-0.75cm}
\texttt{\fbox{
\begin{minipage}{1.05\textwidth}
*MATERIAL, NAME=UMMDp\\
*USER MATERIAL, CONSTANTS=33\\
\textcolor{red}{0}, \textcolor{red}{0}, 200000.0, 0.3, \textcolor{red}{2}, 1.241024, 1.078271, 1.216463, \\
1.223867, 1.093105, 0.889161, 0.501909, 0.557173, 1.349094, 0.775366, 0.922743,\\
0.765487, 0.793356, 0.918689, 1.027625, 1.115833, 1.112273, 0.589787, 8.0,\\
\textcolor{red}{2}, 541.0, 0.0036, 0.249, \textcolor{red}{3}, 1018.4245, 22.85, \textcolor{red}{4},\\
0.5
\end{minipage}
}}
\par\bigskip

\end{document}